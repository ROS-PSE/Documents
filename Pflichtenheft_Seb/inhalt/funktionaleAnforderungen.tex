\chapter{Funktionale Anforderungen}
das sind die von der letzten gruppe 
hab den inhalt mal drinn gelassen damit wir unser zeug einfach nur einf�gen
m�ssen


Erkl�rungen zu den OpenCV spezifischen Begriffen werden im Glossar gegeben.

\section{API \small{\textit{Pflicht}}}

\subsection{Allgemein}
\begin{description}
	\item[/FA0100/] Globale Auswahl zwischen Debug- und Release-Modus
	\item[/FA0200/] Auswahl der Visualisierung f�r jeden Operationstyp
\end{description}

\subsection{Unterst�tzung folgender Operationen von OpenCV}
\begin{description}
		\item[/FA0210/] dilate
		\item[/FA0220/] erode
		\item[/FA0230/] morphologyEx
		\item[/FA0240/] Sobel\\

		\item[/FA0250/] threshold
		\item[/FA0260/] adaptiveThreshold
		\item[/FA0270/] floodFill\\

		\item[/FA0280/] KeyPoint 
		\item[/FA0290/] DMatch
\end{description}


\section{API \small{\textit{Optional}}}

\subsection{Allgemein}
\begin{description}
	\item[/FA0300/] Optionale Parameter f�r Einstellungen der Visualisierungen
	\item[/FA0400/] Lokale Auswahl Debug/Release-Modus
	\item[/FA0500/] Optionale nicht-blockierende Aufrufe f�r Streaming
\end{description}

\subsection{Unterst�tzung folgender Operationen}

\begin{description}
	\item[/FA0600/]  stitching
	\item[/FA0610/]  ocl (OpenCL)\\

 	\item[/FA0620/] calcHist\\

	\item[/FA0630/] Canny
	\item[/FA0640/] HoughCircles
\end{description}


\section{GUI \small{\textit{Pflicht}}}
\begin{description}
	\item[/FA0700/] Eine Visualisierung pro oben gelisteter Operation (siehe API Kriterien)
	\item[/FA0800/] Drei Visualisierungen f�r features2d/DMatch (drei der m�glichen Wunsch-Visualisierungen, siehe unten)
	\item[/FA0900/]  Zoomfunktion
\end{description}

\section{GUI \small{\textit{Optional}}}

\begin{description}
	\item[/FA1000/] Permanente GUI mit Historie
	\item[/FA1200/] M�glichkeit eine Filteroperation mit ge�nderten Parametern erneut anzuwenden
	\item[/FA1300/] Hohe Zoomstufen mit Zusatzinformationen (z.B. Pixelwerte)
	\item[/FA1400/] Optionale Ausnutzung von mehreren Bildschirmen durch Fenstermodus
	\item[/FA1500/] Interaktive �berlagerung der Bilder durch Zusatzinformationen (Mouse over)
	\item[/FA1600/] Flexibler Umgang mit unterschiedlichen Bildschirm- und Bildaufl�sungen
	\item[/FA1700/] Suchleiste f�r alle Tabellen (z.B. jener der �bersichtsseite oder der Rohdatendatenanzeige)
	\item[/FA1710/] Spezielle Syntax zum Beispiel zur Gruppierung von Datens�tzen
\end{description}

\section{M�gliche Visualisierungen}

\subsection{Allgemein}
\begin{description}
	\item[/FA1800/] Darstellung von Rohdaten
	\item[/FA1810/] Abmessungen der Bilder
	\item[/FA1820/] Farbraum der Bilder (der in OpenCV genutzte Datentyp)
	\item[/FA1830/] Tabellarische Darstellung, z.B. der Matches, mit Filterm�glichkeit
	\item[/FA1840/] Diagramme (wie Histogramme)
	\item[/FA1900/] Darstellung der Bilder nebeneinander
\end{description}

\subsection{Visualisierungen von Matches}
\begin{description}
	\item[/FA2010/] Einzeichnen der Keypoints in die Bilder
	\item[/FA2020/] Verbinden der Matches durch Linien oder Pfeile
	\item[/FA2030/] Einf�rben der Linien, Pfeile oder Punkte mit Falschfarben
	\item[/FA2040/] Ausblenden der Keypoints ohne Matches
	\item[/FA2050/] Auswahl von Matches anhand bestimmter Kriterien (z.B. via Histogramm)
	\item[/FA2060/] Manuelle Auswahl von Matches
	\item[/FA2070/] Automatische Zusammenfassung von Matches zu Gruppen\\
	
	\item[/FA2110/] Einzeichnen von Linien / Formen
	\item[/FA2120/] Ausw�hlen von zugeh�rigen Matches
	\item[/FA2130/] Die Linien / Formen werden auf das zweite Bild projiziert
	\item[/FA2140/] Automatische Gruppierung der Matches zu Fl�chen\\
	
	\item[/FA2210/] Pfeill�nge und Richtung entsprechen der jeweiligenTranslation\\
	
	\item[/FA2310/] Pixelfarbwerte entsprechen den jeweiligen Tiefenwerten
\end{description}

\subsection{Visualisierungen f�r Filter}

\begin{description}
	\item[/FA2400/] Differenzbilder
	\item[/FA2500/] �berlagerungen
	\item[/FA2600/]  Direkte Anwendung von Filtern auf ein oder zwei Bilder 
		Beispiel: Anwendung eines Kantenfilters um die Auswirkungen z.B. einer Kantengl�ttung zu visualisieren
	\item[/FA2700/] Visualisierung �ber die Auswirkungen auf bestimmte Bildmetriken 
		Beispiel: �berlagerung von Histogrammen beider Bilder oder Vergleich der Kontrastwerte von bestimmten Bildbereichen
\end{description}


