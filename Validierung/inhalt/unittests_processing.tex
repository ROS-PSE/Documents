\subsection*{TestLoadingSpecifications}

\begin{unittest}{test\_no\_specifications}[pass]
	Der Specificationhandler ist am Anfang leer, wenn er keine Spezifikationen im Namespace findet.
\end{unittest}
\begin{unittest}{test\_load\_spec}[pass]
	Testet, ob eine Spezifikation mit der angegebenen SEUID nach dem Laden vorhanden ist.\\
	Verwendet das format \texttt{[seuid1:{}]}.
\end{unittest}
\begin{unittest}{test\_load\_new\_specs}[pass]
	�berpr�ft, dass auch das format \texttt{section: [seuid1:{}]} geladen wird.
\end{unittest}
\begin{unittest}{test\_reload\_spec}[pass]
	�berpr�ft, dass Spezifikationen zu SEUIDs in den Bestand geladen wurden, nachdem sie auf den Parameterserver geladen und die reload-Methode aufgerufen wurde.
\end{unittest}
\begin{unittest}{test\_invalid\_seuid}[pass]
	�berpr�ft, dass keine Spezifikationen geladen werden, die keine gültige SEUID haben.
\end{unittest}
\begin{unittest}{test\_existing\_fields}[pass]
	�berpr�ft, alle Felder der Definition durch den Parameterserver gekommen sind und im Specification Objekt vorhanden sind.
\end{unittest}