\subsection*{TestLoadingSpecifications}

\begin{unittest}{test\_no\_specifications}[pass]
	Der Specificationhandler ist am Anfang leer, wenn er keine Spezifikationen im Namespace findet.
\end{unittest}
\begin{unittest}{test\_load\_spec}[pass]
	Testet, ob eine Spezifikation mit der angegebenen SEUID nach dem Laden vorhanden ist.\\
	Verwendet das format \texttt{[seuid1:{}]}.
\end{unittest}
\begin{unittest}{test\_load\_new\_specs}[pass]
	�berpr�ft, dass auch das format \texttt{section: [seuid1:{}]} geladen wird.
\end{unittest}
\begin{unittest}{test\_reload\_spec}[pass]
	�berpr�ft, dass Spezifikationen zu SEUIDs in den Bestand geladen wurden, nachdem sie auf den Parameterserver geladen und die reload-Methode aufgerufen wurde.
\end{unittest}
\begin{unittest}{test\_invalid\_seuid}[pass]
	�berpr�ft, dass keine Spezifikationen geladen werden, die keine g�ltige SEUID haben.
\end{unittest}
\begin{unittest}{test\_existing\_fields}[pass]
	�berpr�ft, alle Felder der Definition durch den Parameterserver gekommen sind und im Specification Objekt vorhanden sind.
\end{unittest}


\subsection*{TestRatingData}

\begin{unittest}{test\_no\_data}[pass]
	Wird versucht None zu bewerten, wird None als Verleichsergebnis zur�ckgegeben.
\end{unittest}
\begin{unittest}{test\_invalid\_ident}[pass]
	Ist der Identifier der Eingabedaten ung�ltig, wird None als Verleichsergebnis zur�ckgegeben.
\end{unittest}
\begin{unittest}{test\_no\_ident}[pass]
	Wird kein Identifier mitgegeben, wird None als Verleichsergebnis zur�ckgegeben.
\end{unittest}
\begin{unittest}{test\_no\_spec}[pass]
	Ist keine Spezifikation f�r die angegebene Nachricht geladen, haben die bewerteten Felder den Status 2 (Unknown).
\end{unittest}
\begin{unittest}{test\_spec3}[pass]
	Werden die Spezifikationen erf�llt, wird f�r die jeweiligen Felder der Status 3 (OK) zur�ckgegeben.
\end{unittest}
\begin{unittest}{test\_spec2}[pass]
	Sind einzelne Felder nicht spezifiziert, wird f�r diese der Status 2 (Unknown) zur�ckgegeben.
\end{unittest}
\begin{unittest}{test\_spec10}[pass]
	Felder, die Werte �ber ihren Limits liefern, werden mit dem Status 0 (High) bewertet.\\
	Felder, die Werte unter ihren Limits liefern, werden mit dem Status 1 (Low) bewertet.
\end{unittest}