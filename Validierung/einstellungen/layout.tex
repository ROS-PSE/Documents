%Hier sind alle Einstellungen enthalten, die sich auf das Seiten- und Dokumentenlayout beziehen

\documentclass[
	11pt,								% Schriftgr��e
	DIV12,
	german,							% f�r Umlaute, Silbentrennung etc.
	oneside,						% einseitiges Dokument
	titlepage,					% es wird eine Titelseite verwendet
	halfparskip,				% Abstand zwischen Abs�tzen (halbe Zeile)
	normalheadings,			% Gr��e der �berschriften verkleinern
	tablecaptionabove,	% Beschriftung von Tabellen unterhalb ausgeben
	final								% Status des Dokuments (final/draft)
]{scrreprt}						% 

\makeatletter
\def\@makechapterhead#1{%
  %%%%\vspace*{50\p@}% %%% removed!
  { \parindent \z@ \raggedright \normalfont
    \ifnum \c@secnumdepth >\m@ne
        \Huge\bfseries  \thechapter
        \space\nobreak
    \fi
    \interlinepenalty\@M
    \Huge \bfseries #1\par\nobreak
    \vskip 20\p@
  }}
\def\@makeschapterhead#1{%
  %%%%%\vspace*{50\p@}% %%% removed!
  {\parindent \z@ \raggedright
    \normalfont
    \interlinepenalty\@M
    \Huge \bfseries  #1\par\nobreak
    \vskip 20\p@
  }}
\makeatother

%------�ndern von Schriftschnitten - (Muss ganz am Anfang stehen !) -------------
\usepackage{fix-cm}

%------Umlaute ------------------------------------------------------------------
% 	Umlaute/Sonderzeichen wie ���� k�nnen direkt im Quelltext verwenden werden.
%		Erlaubt automatische Trennung von Worten mit Umlauten.
\usepackage[T1]{fontenc}								 
\usepackage[latin1]{inputenc}

%------Anpassung der Landessprache-----------------------------------------------
\usepackage{ngerman}

%------Einfache Definition der Zeilenabst�nde und Seitenr�nder-------------------
\usepackage{geometry}
\usepackage{setspace}
\usepackage{tocbasic}

%------Schriftgr��enanpassung von einzelnen Textpassagen-------------------------
\usepackage{relsize}

%------Trennlinien in Kopf- und Fusszeile
\usepackage[headsepline, footsepline, ilines]{scrpage2}

%------Grafiken------------------------------------------------------------------
\usepackage{graphicx}
\usepackage{float}

%------Packet zum Sperren, Unterstreichen und Hervorheben von Texten------------
\usepackage{soul}

%------erg�nzende Schriftart----------------------------------------------------
\usepackage{helvet}

%------Lange Tabellen-----------------------------------------------------------
\usepackage{longtable}
\usepackage{array}
\usepackage{ragged2e}
\usepackage{lscape}

\usepackage{xparse}
\usepackage{framed}
\usepackage[usenames,dvipsnames]{color}

%------PDF-Optionen-------------------------------------------------------------
\usepackage[
	bookmarks,
	bookmarksopen=true,
	colorlinks=true,
	linkcolor=black,				% einfache interne Verkn�pfungen
	anchorcolor=black,			% Ankertext
	citecolor=black, 				% Verweise auf Literaturverzeichniseintr�ge im Text
	filecolor=black, 				% Verkn�pfungen, die lokale Dateien �ffnen
	menucolor=black, 				% Acrobat-Men�punkte
	urlcolor=black, 				% Farbe f�r URL-Links
	backref,								% Zur�cktext nach jedem Bibliografie-Eintrag als Liste von �berschriftsnummern
	pagebackref,						% Zur�cktext nach jedem Bibliografie-Eintrag als Liste von Seitenzahlen
	plainpages=false,				% zur korrekten Erstellung der Bookmarks
	pdfpagelabels,					% zur korrekten Erstellung der Bookmarks
	hypertexnames=false,		% zur korrekten Erstellung der Bookmarks
	% linktocpage 						% Seitenzahlen anstatt Text im Inhaltsverzeichnis verlinken
	]{hyperref}



			% enth�lt eingebundene Packete

%------Seitenr�nder-------------------------------------------------------------
\geometry{verbose, 										% zeigt die eingestellten Parameter beim Latexlauf an
			paper=a4paper, 									% Papierformat			
			top=25mm, 											% Rand oben
			left=25mm, 											% Rand links
			right=25mm, 										% Rand rechts
			bottom=45mm, 										% Rand unten
			pdftex													% schreibt das Papierformat in dei Ausgabe damit Ausgabeprogramm Papiergr��e erkennt		
	} 
	
%Seitenlayout
\onehalfspace        % 1,5-facher Abstand  

%------Kopf- und Fu�zeilen ------------------------------------------------------
\pagestyle{scrheadings}

%------Kopf- und Fu�zeile auch auf Kapitelanfangsseiten -------------------------
\renewcommand*{\chapterpagestyle}{scrheadings}

%------Schriftform der Kopfzeile ------------------------------------------------
\renewcommand{\headfont}{\normalfont}

%------Kopfzeile-----------------------------------------------------------------
\setlength{\headheight}{21mm}				% H�he der Kopfzeile
\ihead{\large{\textsc{\praktikumTitel}}\\		% Text in der linken Box
			 \small{\projektTitel}}
\chead{}														% Text in der mittleren Box

%----Fusszeile
\cfoot{}														% Text in mittlerer Box
\ofoot{\pagemark}										% Seitenzahl in rechter Box			


% \begin{scetch}{title}{img file}
\newenvironment{sketch}[2][\skip]{
	\subsection{#1}
	\sketchImg{#1}{#2}	
	\label{#2}
	
}

\newcommand{\sketchref}[1]{
	\textsc{\nameref{#1}}
}

\newcommand{\glossaryItem}[2]{
	\item[#1] \label{#1} #2
}

% \scetchImg{title}{file name}
\newcommand{\sketchImg}[2]{
	\begin{figure}[H]
		\centering
		\includegraphics[width=\linewidth]{Bilder/#2}
		\caption{#1}
	\end{figure}
}

