%------�ndern von Schriftschnitten - (Muss ganz am Anfang stehen !) -------------
\usepackage{fix-cm}

%------Umlaute ------------------------------------------------------------------
% 	Umlaute/Sonderzeichen wie ���� k�nnen direkt im Quelltext verwenden werden.
%		Erlaubt automatische Trennung von Worten mit Umlauten.
\usepackage[T1]{fontenc}								 
\usepackage[latin1]{inputenc}

%------Anpassung der Landessprache-----------------------------------------------
\usepackage{ngerman}

%------Einfache Definition der Zeilenabst�nde und Seitenr�nder-------------------
\usepackage{geometry}
\usepackage{setspace}

%------Schriftgr��enanpassung von einzelnen Textpassagen-------------------------
\usepackage{relsize}

%------Trennlinien in Kopf- und Fusszeile
\usepackage[headsepline, footsepline, ilines]{scrpage2}

%------Grafiken------------------------------------------------------------------
\usepackage{graphicx}
\usepackage{float}

%------Packet zum Sperren, Unterstreichen und Hervorheben von Texten------------
\usepackage{soul}

%------erg�nzende Schriftart----------------------------------------------------
\usepackage{helvet}

%------Lange Tabellen-----------------------------------------------------------
\usepackage{longtable}
\usepackage{array}
\usepackage{ragged2e}
\usepackage{lscape}

%------PDF-Optionen-------------------------------------------------------------
\usepackage[
	bookmarks,
	bookmarksopen=true,
	colorlinks=true,
	linkcolor=black,				% einfache interne Verkn�pfungen
	anchorcolor=black,			% Ankertext
	citecolor=black, 				% Verweise auf Literaturverzeichniseintr�ge im Text
	filecolor=black, 				% Verkn�pfungen, die lokale Dateien �ffnen
	menucolor=black, 				% Acrobat-Men�punkte
	urlcolor=black, 				% Farbe f�r URL-Links
	backref,								% Zur�cktext nach jedem Bibliografie-Eintrag als Liste von �berschriftsnummern
	pagebackref,						% Zur�cktext nach jedem Bibliografie-Eintrag als Liste von Seitenzahlen
	plainpages=false,				% zur korrekten Erstellung der Bookmarks
	pdfpagelabels,					% zur korrekten Erstellung der Bookmarks
	hypertexnames=false,		% zur korrekten Erstellung der Bookmarks
	% linktocpage 						% Seitenzahlen anstatt Text im Inhaltsverzeichnis verlinken
	]{hyperref}



