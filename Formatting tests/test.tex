\documentclass[10pt,a4paper]{article}
\usepackage[utf8]{inputenc}
\usepackage{amsmath}
\usepackage{amsfonts}
\usepackage{amssymb}
\begin{document}
\section{mQuery - Framework for
Manialinks}\label{mquery---framework-for-manialinks}

\subsection{What it is}\label{what-it-is}

mQuery included the possiblitiy of formating your ManiaPlanet Manialink
with attributes like you are used to from css. Usualy these are the
normal Manialink element's attributes, but there are even more
possibilities automating even more. The other point in mQuery is the
jQuery like mQuery language giving you huge functionality in very few
commands, saving you a lot of time and worries.

\subsection{How it works}\label{how-it-works}

The framework preprocesses your input, excludes unnecessary parts that
were only used to implement your mQuery. It automatically adds necessary
Manialink elements and adds the css formatings according to id- and
classnotations.

\subsection{How to use it}\label{how-to-use-it}

First of all you need to catch your manialink, for example with

\begin{Shaded}
\begin{Highlighting}[]
\KeywordTok{<?php}  
\FunctionTok{ob_start}\OtherTok{();}  
\KeywordTok{include}\OtherTok{(}\StringTok{'manialink.xml'}\OtherTok{);}  
\KeywordTok{$MLData} \NormalTok{= }\FunctionTok{ob_get_contents}\OtherTok{();}    
\FunctionTok{ob_end_clean}\OtherTok{();}  
\KeywordTok{$MLDoc} \NormalTok{= }\KeywordTok{new} \NormalTok{ManialinkAnalysis\textbackslash{}ManialinkAnalizer}\OtherTok{(}\KeywordTok{$MLData}\OtherTok{);}  \CommentTok{// Preprocess the manialink  }
\KeywordTok{$MLDoc}\NormalTok{->applyStyles}\OtherTok{();}                                      \CommentTok{// Apply the css styles to the manialink elements  }
\KeywordTok{$MLDoc}\NormalTok{->output}\OtherTok{();}                                           \CommentTok{// Output formatted manialink }
\KeywordTok{?>}
\end{Highlighting}
\end{Shaded}

In the manialink you can declare stylesheet files with
\texttt{\textless{}style href="*{[}filename.css{]}*" /\textgreater{}}.
mQuery code files can be included with
\texttt{\textless{}script src="*{[}scriptfile.whatever{]}*" /\textgreater{}}
or written within
\texttt{\textless{}mquery\textgreater{}\textless{}/mquery\textgreater{}}blocks.

\subsection{Version}\label{version}

0.4 alpha
\end{document}