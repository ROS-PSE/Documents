\chapter{Systemmodell}

\section{Szenarien}

\subsection*{Definition des Soll-Zustandes}
\begin{itemize}
	\item Der Benutzer m�chte entscheiden, welche Daten �berwacht werden sollen.
	\item Er definiert eine Datei, welche den Namen und Typ der zu speichernden Daten beschreibt.
	\item Er bestimmt einen Bereich, in dem die Werte liegen sollen.
	\item Die Metadaten werden �ber Publisher-Subscriber an einen Monitor-Knoten geschickt.
\end{itemize}

\subsection*{Darstellung von Fehlern}
\subsubsection*{Fehlerfeststellung durch den Benutzer}
\begin{itemize}
	\item Einem Benutzer ist ein Fehler aufgefallen: Der Roboter verh�lt sich nicht
	mehr, wie es erwartet wird.
	\item Er �ffnet das PSE Introspection Plugin in der rqt GUI.
	\item Sofort sieht er, dass das Plugin den Fehler erfasst hat. Es zeigt ein Problem an.
	\item In einer Knoten�bersicht f�llt es dem Benutzer leicht, die farblich
	auff�llig hinterlegten, fehlerhaften Knoten auszumachen.
	\item Er �ffnet die betroffenen Knoten in der Detailansicht und sieht, in welchen Punkten das Knotenverhalten von den Spezifikationen abweicht.
\end{itemize}
\subsubsection*{Fehlerfeststellung �ber die GUI}
\begin{itemize}
	\item Um den Status des ROS-Netzwerkes im Auge zu behalten, hat der Benutzer
	das PSE Introspection Plugin in der rqt GUI ge�ffnet.
	\item Er stellt fest, dass die Statusanzeige nicht mehr in Gr�n volle Funktionalit�t signalisiert.
	\item Mit einem Klick �ffnet er das Fehlerlog, in dem er Eintr�ge findet, welche Knoten von ihren Spezifikationen abweichend arbeiten und die Fehlermeldung ausgel�st haben.
	\item Nun hat der Anwender die M�glichkeit, die Fehlerhaften Knoten aus einer Liste zu suchen und sie in einer Detailansicht zu �ffnen, um sich weitere Details �ber die ausgefallenen Komponenten anzeigen zu lassen.
\end{itemize}

\subsection*{Knotendetail-Ansicht}
\subsubsection*{Auswahl eines Knotens}
\begin{itemize}
	\item Der Benutzer m�chte sich die Metadaten Informationen zu einen Knoten anzeigen.
	\item Er w�hlt sich den Knoten, der ihn interessiert, aus einer Liste.
	\item Dazu besteht die M�glichkeit, diesen in einer Liste zu finden, welche sich auch �ber ein Suchfeld filtern l�sst.
	\item W�hlt er einen Knoten aus, �ffnet sich ein kleines Fenster mit den Knoteninformationen und reiht sich in die �bersicht der ge�ffneten Detailfenster ein.
\end{itemize}
\subsubsection*{Umschalten der Detailansicht am Beispiel der Topic Liste}
\begin{itemize}
	\item Das Detailfenster bietet f�r verschiedene Knotendaten Links, um die Ansicht zu erweitern.
	\item Der Benutzer bekommt ein Topic angezeigt, mit dem der Knoten in einem
	Publisher-/Subscriber-Verh�ltnis steht.
	\item Klickt der Benutzer auf eine Schaltfl�che neben der Anzeige, wird die
	Anzeige der sonstigen Metadaten durch eine vollst�ndige Auflistung der relevanten Topics ersetzt.
	\item Intuitiv positioniert findet der Anwender eine Schaltfl�che, um zur vorigen Ansicht zur�ckzukehren.
\end{itemize}

\subsection*{Dynamische GUI}
\begin{itemize}
	\item Der Benutzer schlie�t ein rqt Plugin, das er nicht mehr ben�tigt.
	\item Daraufhin verschiebt sich das Layout in der rqt GUI und das PSE Introspection Plugin �ndert seine Form und Gr��e.
	\item Das PSE Introspection Plugin rearrangiert seine Inhalte, um sie im neuen Layout anzuzeigen.
\end{itemize}

\section{Anwendungsf�lle}

\subsection*{�berwachung eines Knoten}
\begin{itemize}
	\item Primary Actor: Monitor
	\item Level: User Goal
	\item Scope: Monitor System
\end{itemize}
\subsubsection*{Story}
 Der Benutzer definiert einen Soll-Zustand f�r den Knoten und stellt eine
 Verbindung zum Monitor her.
		  Der zu �berwachende Knoten sendet seinen Ist-Zustand an den Monitoring-Knoten.
		  Der Monitor-Knoten gleicht den Ist-Zustand mit dem vorher definierten
		  Soll-Zustand ab.
		  Falls eine Abweichung entdeckt wird, zeigt das rqt Widget den betroffenen
		  Knoten an.
		  Der Benutzer kann sich f�r eine Gegenma�nahme entscheiden.
\subsubsection*{Postcondition}
\begin{itemize}
	\item Fehlerzustand wurde erkannt
	\item Angemessene Gegenma�nahmen konnten getroffen werden.
\end{itemize}
\subsubsection*{Main Succes Scenario}
\begin{enumerate}
	\item Benutzer definiert Soll-Zustand
	\item Verbindung zwischen Knoten und Monitor wird hergestellt
	\item Monitor empf�ngt Meta-Daten
	\item Monitor gleicht den Soll-Zustand mit dem Ist-Zustand ab
	\item Diskrepanz wird festgellt
	\item Fehlerhafter Knoten wird auf GUI angezeigt
	\item Benutzer kann Gegenma�nahme treffen
\end{enumerate}
\subsubsection*{Frequency of Occurrence}
 Durchl�ufig angewendet




