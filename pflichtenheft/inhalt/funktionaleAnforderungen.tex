\newcounter{fac}
\newcounter{facsec}
\newcounter{temp}
\setcounter{fac}{0}
\setcounter{facsec}{0}

\newcommand{\fa}
{
\addtocounter{fac}{100}
\setcounter{facsec}{0}

\ifnum\value{fac}<1000
\item[/FA0\thefac/]
\else
\item[/FA\thefac/]
\fi
}

\newcommand{\fasec}
{
\addtocounter{facsec}{10}
\setcounter{temp}{\thefac}
\addtocounter{temp}{\thefacsec}
\ifnum\value{fac}<1000
\item[/FA0\thetemp/]
\else
\item[/FA\thetemp/]
\fi


}

\chapter{Funktionale Anforderungen}

\section{Gesamtsystem}
\subsection{Pflicht}
\begin{description}
\fa Dezentrale Erfassung von Metadaten: Anzahl Topics und Subscriber, Bandbreite, Frequenz, Latenz, Jitter
\fa Monitoring Knoten zum zentralen Abgleich des Soll- und Ist-Zustandes
\fa Countermeasure Knoten um Gegenma�nahmen bei Fehlern ergreifen zu k�nnen
\fasec Warnungen und Fehlernachrichten bei ma�geblichen Abweichungen
\fa Definition eines ROS Message Types f�r Metadaten
\fa Eigenst�ndiger Knoten zur �berwachung der Hardware des Host-Systems (CPU Auslastung, CPU Temperatur, RAM, Festplatten-Speicher)
\end{description}

\subsection{Optional}
\begin{description}
\fa Lebenszeichen von Subscribern und Hosts, auch wenn keine Daten Empfangen werden
\fa �berwachung weiterer ROS Komponenten wie Services und Parameters
\fa �berwachung der Publisher
\fasec Erfassung der Anzahl Publisher pro Topic
\fasec Erfassung von Lebenszeichen von Publishern
\fa Festlegen des Ist-Zustandes als Soll-Definition
\fasec Korrelationen zwischen Empfangs- und Sendeverhalten eines Knotens erkennen
\fa Anpassung des Sendeverhaltens des Systems an Netzwerkgegebenheiten
\fa Definiton der Netzwerktopologie sowie darstellung der Auslastung physischer Verbindungen
\fa Integration mit roswtf
\end{description}

\section{Soll-Spezifikation}
\subsection{Pflicht}
\begin{description}
\fa Parametriesierung f�r Topics, Hosts und Knoten getrennt
\fasec Obere und untere Schranken f�r Metadaten definierbar
\fa Laden der Soll-Spezifikation einmalig bei Start des �berwachungsknotens
\fasec Weitergabe der Soll-Spezifikation an den ROS Parameter Server bei Knotenstart
\fa M�glichkeit nur Teile des Netzwerkes zu �berwachen
\fasec Teilsysteme k�nnen sich �berlappen
\end{description}

\subsection{Optional}
\begin{description}
\fa Ein Monitoringknoten verwaltet alle Teilsysteme
\fa Gegema�nahmen in der Soll Spezifikation definierbar machen
\end{description}
\section{API}
\subsection{Pflicht}
\begin{description}
\fa Die Metadaten werden durch Hinzuf�gen eines Funktionsaufrufes zu bestehenden Callbacks erfasst, alternativ durch eine geeignete Klasse, die von Subscriber erbt
\fa Die Metadaten werden auf einem Topic mit definierter Frequenz publiziert
\fa Die Metadatenerfassung l�sst sich �ber Parameter deaktivieren
\end{description}


\subsection{Optional}
\begin{description}
\fa Knoten sind in der Lage, sich anhand ihrer Soll-Metadaten selber zu �berwachen
\fa Das System wird auch f�r C++ Knoten implementiert
\end{description}

\section{GUI}
\subsection{Pflicht}
\begin{description}
\fa Die grafische Benutzeroberfl�che bietet eine Visualisierung des Soll-Ist-Vergleichs von sowohl Knoten als auch Hostsystemen
\end{description}


\subsection{Optional}
\begin{description}
\fa Geeignete Metadaten werden in nahezu Echtzeit im Knoten-Graphen visualisiert (verf�gbare Bandbreite, Datenrate, Latenz)
\fa Der Soll-Ist-Vergleich wird in einer Visualisierung des ROS Graphen dargestellt
\fa Geeignete Metadaten werden im ROS Graphen dargestellt
\fa Es wird ein zeitlicher Verlauf der Metadaten aufgezeichnet und grafisch dargestellt
\fa Knoten werden im Graphen nach ihrem Host gruppiert angezeigt
\fa Knoten werden als Gruppen der unterschiedlichen Soll-Spezifikationen angezeigt
\end{description}

