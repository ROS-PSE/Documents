\newcounter{fac}
\newcounter{facsec}
\newcounter{temp}
\setcounter{fac}{0}
\setcounter{facsec}{0}

\newcommand{\fa}
{
\addtocounter{fac}{100}
\setcounter{facsec}{0}

\ifnum\value{fac}<1000
\item[/FA0\thefac/]
\else
\item[/FA\thefac/]
\fi
}

\newcommand{\fasec}
{
\addtocounter{facsec}{10}
\setcounter{temp}{\thefac}
\addtocounter{temp}{\thefacsec}
\ifnum\value{fac}<1000
\item[/FA0\thetemp/]
\else
\item[/FA\thetemp/]
\fi


}

\chapter{Funktionale Anforderungen}

\section{Gesamtsystem}
\subsection{Pflicht}
\begin{description}
\fa Dezentrale Erfassung von Metadaten: Anzahl Publisher und Subscriber, Bandbreite, Frequenz, Latenz, Jitter
\fa Modulare Definition der Metadaten anhand von Metadaten und dem ROS-Graph
\fa Knoten zum zentralen Abgleich des Soll- und Ist-Zustandes
\fasec Warnungen und Fehlernachrichten bei ma�geblichen Abweichungen
\fa Definition eines ROS Message Types f�r Metadaten
\fa Eigenst�ndiger Knoten zur �berwachung der Hardware des Host-Systems (CPU Auslastung, CPU Temperatur, RAM, Festplatten-Speicher)
\end{description}

\subsection{Optional}
\begin{description}
\fa Eigenst�ndiger Knoten zur �berwachung der Hardware des Host-Systems
\fa �berwachung weiterer ROS Komponenten wie Services und Parameters
\fa Definition und �berwachung des Empfangs- und Sendeverhaltens eines Knotens
\fa Festlegen des Ist-Zustandes als Soll-Definition
\fa Anpassung des Systems ans Netzwerkgegebenheiten
\fa Integration mit roswtf
\end{description}

%todo: an richtige stelle schieben und anpassen
\subsection{Spezifikation}
\begin{description}
\fa Spezifikationen in yaml speichern
\fa Parametriesierung f�r Topics, Hosts und Knoten getrennt
\fasec Bereichangabe als Soll Spezifikation
\fa Defaultwerte f�r Spezifikationsparameter
\fa �bergabe der Spezifikation bei Start des �berwachungsknotens
\fasec Weitergabe der Soll Spezifikation an den ROS Parameter Server bei Knotenstart
\fa Bestimmte Teilsysteme �berwachen
\fasec Teilsysteme k�nnen sich �berlappen
\fa Ein �berwachungsknoten f�r alle Teilsysteme
\end{description}

\section{API}
\subsection{Pflicht}
\begin{description}
\fa Die Metadaten werden durch Hinzuf�gen eines Funktionsaufrufes zu bestehenden Callbacks erfasst
\fa Die Metadaten werden auf einem Topic mit definierter Frequenz publiziert
\fa Die Metadatenerfassung l�sst sich �ber Parameter deaktivieren
\end{description}


\subsection{Optional}
\begin{description}
\fa Knoten sind in der Lage, sich anhand ihrer Soll-Metadaten selber zu �berwachen
\fa Das System wird auch C++-seitig implementiert
\end{description}

\section{GUI}
\subsection{Pflicht}
\begin{description}
\fa Die grafische Benutzeroberfl�che bietet eine Visualisierung des Soll-Ist-Vergleichs von sowohl Knoten als auch Hostsystemen
\end{description}


\subsection{Optional}
\begin{description}
\fa Der Soll-Ist-Vergleich wird in einer Visualisierung des ROS Graphen dargestellt
\fa Es wird ein zeitlicher Verlauf der Metadaten aufgezeichnet und grafisch dargestellt
\fa Knoten werden im Graphen nach ihrem Host gruppiert angezeigt
Unsere Plugins sind in der Lage untereinander zu kommunizieren
\end{description}

