\chapter{Glossar}
\label{sec:glossar}
\section{Allgemein}
\begin{description}
 \glossaryItem{API}{Application Programming Interface. Eine Schnittstelle, �ber die Programme auf Funktionalit�t anderer Programme zugreifen k�nnen}
 \glossaryItem{CPU}{Central Processing Unit. Das englische Akronym f�r den Prozessor eine Computers}
 \glossaryItem{GUI}{Graphische Benuteroberfl�che, original aus dem Englischen: Graphical User Interface}
 \glossaryItem{Middleware}{Schicht, die zwischen Anwendungen und der Infrastruktur vermittelt um so die Komplexit�t der anderen Schichten zu verbergen}
 \glossaryItem{Open Source}{Software die Quelloffen ist und weiterentwickelt sowie weiterverbreitet werden darf} 
 
\end{description}
\section{ROS}
\begin{description}
 \glossaryItem{Knoten}{Siehe Node}
 \glossaryItem{Master}{Der Master erm�glicht die Namensaufl�sung/-registrierung und ist zust�ndig f�r die Verkn�pfung von Knoten miteinander}
 \glossaryItem{Message}{Nachricht, welche zwischen Knoten versendet wird. Eine Nachricht besteht aus primitiven Datentypen} 
 \glossaryItem{Node}{Auch Knoten genannt. Repr�sentiert ein Prozess. Erm�glicht verschiedene Aufgaben zu trennen und so das ROS Netzwerk modular zu gestalten.}
 \glossaryItem{Parameter Server}{Der Parameter Server erlaubt das zentrale Speichern von Daten. Alle Knoten k�nnen auf ihn zugreifen}
 \glossaryItem{Publisher}{Objekt, das Messages unter bestimmten Topics ver�ffentlicht}
 \glossaryItem{ROS}{Das Robot Operating System ist eine Open-Source Middleware um flexible Robotik Software zu schreiben}
 \glossaryItem{Service}{Services dienen zur zweiwege Kommunikation in ROS Netzwerken, da das Publisher/Subscriber System nur einwege Kommunikation unterst�tzt. Services verhalten sich wie entfernte Funktionsaufrufe (RPC)}
 \glossaryItem{Subscriber}{Objekt, das Messages von abonnierten Topics empf�ngt }
 \glossaryItem{Topic}{Ein Topic ist ein Name um den Inhalt einer Message zu definieren. Publisher k�nnen Messages auf bestimmten Topics publizieren, Subscriber k�nnen Messages von abonnierten Topics erhalten}
\end{description}
