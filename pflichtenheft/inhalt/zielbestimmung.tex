\chapter{Zielbestimmung}

Die zu entwickelnde Softwarel�sung soll sich als �berwachungsinstrument in die
Robot Operating System (ROS) Middleware einf�gen. Dabei soll �berwacht werden, ob jeder Prozess, der im ROS-Netzwerk enthalten ist, definierten Sollwerten entsprechend, ordnungsgem�� Daten �bermittelt.\\
Gerade durch die gegebene Problematik, eines auf mehrere Hosts verteilten
Systems, bereitet das Erfassen und Verstehen von auftretenden Problemen, zum
aktuellen Stand des ROS, Schwierigkeiten. Das vorliegende Projekt erfasst den
Ist-Zustand der Prozesse im ROS-Netzwerk, ohne Leistungseinbu�en, kontinuierlich
und visualisiert diesen im Vergleich zu definierten Soll-Werten. Auftretende
Fehler und Abweichungen von den Spezifikationen, werden dem Benutzer deutlich
angezeigt. Es lassen sich Gegenma�nahmen f�r verschiedene Fehler definieren.\\

\section{Musskriterien}
\begin{itemize}
	\item Minimale Erweiterung bestehender Knoten zur �bermittlung von Metadaten
	\item Publizierung der Metadaten mit defnierter Frequenz auf Topics
	\item Deaktivierung der Datenerfassung einzelner Knoten
	\item Definition von Soll-Zust�nden
	\item Graphische �bersicht �ber Abweichungen von Soll- und Ist-Zustand der Knotendaten
	\item Statuserfassung der Host-Computer
\end{itemize}

\section{Wunschkriterien}
\begin{itemize}
	\item �berwachung weiterer ROS Bestandteile: Service, Parameters
	\item Selbstkontrolle der Knoten
	\item Ist-Zustand als Soll-Zustand definieren
	\item Erg�nzende �bersichten mit dem ROS Graphen und Plots �ber den zeitlichen Verlauf
\end{itemize}

\section{Abgrenzungskriterien}
\begin{itemize}
	\item Das Projekt wird keine vollst�ndige Automatisierung in Fehlerf�llen umsetzen.
	\item Es wird keine vollst�ndige �berwachung der Netzwerkinfrastruktur
	stattfinden.
	\item Die Metadatenerfassung umfasst nur Knoten die daf�r modifiziert werden.
\end{itemize}
