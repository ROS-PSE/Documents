\newcounter{nfc}
\newcounter{nfcsec}
\newcounter{tempsec}
\setcounter{nfc}{0}
\setcounter{nfcsec}{0}

\newcommand{\nf}
{
\addtocounter{nfc}{100}
\setcounter{nfcsec}{0}

\ifnum\value{nfc}<1000
\item[/NF0\thenfc/]
\else
\item[/NF\thenfc/]
\fi
}

\newcommand{\nfsec}
{
\addtocounter{nfcsec}{10}
\setcounter{tempsec}{\thenfc}
\addtocounter{tempsec}{\thenfcsec}
\ifnum\value{nfc}<1000
\item[/NF0\thetempsec/]
\else
\item[/NF\thetempsec/]
\fi
}





\chapter{Nichtfunktionale Anforderungen}
\section{Produktleistungen}
\subsection{Pflicht}
\begin{description}
\nf Einfaches einbinden der Metadatenerfassung in bereits vorhandene Knoten
\nf Geringen zus�tzlichen Leistungsbedarf durch die Metadatenerfassung
\nfsec Vernachl�ssigbarer Leistungsbedarf wenn die Erfassung deaktiviert ist
\nf Stufenweise Selbst�berwachung des Gesamtsystems
\nfsec Selbst�berwachung anhand Knoten und ihre Verbindungen
\nfsec Selbst�berwachung durch die der Metadaten der Knoten
\nfsec Selbst�berwachung unter Ber�cksichtigung von Hosts und Netwerktopologie
\end{description}

\subsection{Optional}
\begin{description}
\nf Flexibles GUI Layout
\end{description}

\section{Qualit�tsanforderungen}
\subsection{Pflicht}
\begin{description}
\nf M�glichkeit die Metadaten zu erweitern
\nf Modularit�t der GUI
\nf Dokumentationen in Englisch
\nfsec Dokumentation der API
\nfsec Dokumentation der GUI 
\nfsec Dokumentation des Python-Codes mittels PyDoc oder Sphinx
\end{description}

\subsection{Optional}
\begin{description}
\nf Tutorial zur Nutzung des Programmes in Englisch
\nf Weitgehende Abdeckung durch Tests
\end{description}
