\newcounter{tfc}
\newcounter{tfcsec}
\newcounter{tmpsec}
\setcounter{tfc}{0}
\setcounter{tfcsec}{0}

\newcommand{\tf}{
	\addtocounter{tfc}{100}
	\setcounter{tfcsec}{0}
	
	\ifnum\value{tfc}<1000
		\item[/TF0\thetfc/]
	\else
		\item[/TF\thetfc/]
	\fi
}


\newcommand{\tfsec}
{
\addtocounter{tfcsec}{10}
\setcounter{tmpsec}{\thetfc}
\addtocounter{tmpsec}{\thetfcsec}
\ifnum\value{tfc}<1000
\item[/TF0\thetmpsec/]
\else
\item[/TF\thetmpsec/]
\fi
}


\chapter{Testf�lle und Testszenarien}
\section{Tests}
% hier eigentlich nur automatische unittest sachen bzw. werden visuelle best�tigungen des verhaltens durch die szenarien abgedeckt?
\begin{description}
	\tf Der Monitoring-Knoten erkennt abweichendes Verhalten und gibt diese Information an den Countermeasure-Knoten weiter.
	\tf Eine Komponente sendet mit festgelegter Frequenz Metadaten. Sie kommen mit der selben Frequenz vollst�ndig am Monitoring-Knoten an.
	% erw�nschtes verhalten? kann sowas �berhaupt auftreten:
	\tf Ein Metadatenpaket �berschreitet auff�llig das Mittel, folgende Pakete halten sich im Rahmen. Das fehlerhafte Paket wird aufgezeichnet aber ignoriert.
	\tf Eingelesene Konfigurationsdaten lassen sich parsen und auslesen.
	\tf Werden keine eigenen Konfigurationsdaten angegeben, werden sinnvolle Standardwerte verwendet.
	\tf Die Hostdaten werden korrekt �bermittelt.
	\tf Optional: Ein Knoten erh�lt Daten von mehreren anderen Knoten. Er informiert, wenn er weniger Quellen hat, als �blich.
	\tf Optional: Funktionsweise des Startens und Stoppens von Knoten unter Netzwerklast pr�fen
	% Nicht automatisiert testbar
	% \tf Die GUI reagiert darauf, wenn ein Knoten ma�geblich von seinen Sollwerten abweicht, indem z.B. der problematische Knoten eingef�rbt wird.
	% \tf rqt Plugins reagieren korrekt auf Gro�en�nderung
	% \tf Menge der Metadaten �berschreitet nicht die Netzwerkkapazit�t
\end{description}

\section{Beispielszenarien}

\subsection*{Typische Vorgehensweise}
\begin{itemize}
	\item Der Benutzer startet zun�chst roscore, dann neben den von ihm ben�tigten Knoten den Monitoring Knoten.
	\item Um den aktuellen Status seiner Knoten und des Netzwerks zu �berwachen, startet er nun rqt und �ffnet das Listen Widget. Falls nun Fehler auftreten, wird er durch eine farbliche Ver�nderung auf m�gliche Probleme hingewiesen.
\end{itemize}

\subsection*{Knoten exisitert nicht mehr}
\begin{itemize}
  \item Ein ROS-Netzwerk mit beliebig vielen Knoten wird gestartet.
  \item Der Benutzer �ffnet das Plugin in der rqt GUI.
  \item Nun wird ein Knoten vom Netzwerk getrennt (z.B. durch Programmabsturz).
  \item	Falls dieser Knoten in den Soll-Spezifkationen als wichtig definiert wurde, reagiert die GUI nun darauf und zeigt an, dass der Knoten nicht mehr
  verf�gbar ist und dass ein Fehler aufgetreten ist.
  \item Optional bietet die GUI nun die M�glichkeit den betroffenen Knoten neuzustarten.
\end{itemize}

\subsection*{Knoten weicht von seinen Sollwerten ab}
\begin{itemize}
  \item Wieder wird ein ROS-Netzwerk mit beliebig vielen Knoten gestartet.
  \item Es werden die Soll Spezifikationen festgelegt und der Monitoring Knoten gestartet.
  \item Der Zustand eines Knotens verschlechtert sich gravierend und er
  �berschreitet die Soll-Spezifikation an mindestens einem Punkt.
  \item Der Monitoring Knoten erkennt das �berschreiten der Spezifikation und startet ggf. vordefinierte Aktionen.
  \item Der Benutzer �ffnet den Node Graph in der GUI und sieht den betreffenden Knoten aufgrund der farblichen Hervorhebung sofort. Optional kann er sich auch in Form eines Graphen den Verlauf der Daten �ber bestimmte Zeitr�ume hinweg anzeigen lassen.
  \item Der Benutzer hat nun entweder die M�glichkeit den Knoten aus der GUI heraus neuzustarten oder er nimmt manuell Reparaturen vor.
\end{itemize}

\subsection*{Countermeasure Knoten}
\begin{itemize}
  \item Vorrausgesetzt wird das vorherige Szenario, in dem ein Knoten die
  Soll-Spezifikation �berschreitet.
  \item Nun reagiert der Countermeasure Knoten auf diese Abweichung und leitet
  vordefinierte Gegenma�nahmen ein.
  \item Z.B. wird der Name des betroffenen Knotens in der Shell
  ausgegeben.
\end{itemize}

\subsection*{Dynamische GUI}
\begin{itemize}
	\item Der Benutzer schlie�t ein rqt Plugin, das er nicht mehr ben�tigt.
	\item Daraufhin verschiebt sich das Layout in der rqt GUI und das PSE Introspection Plugin �ndert seine Form und Gr��e.
	\item Das PSE Introspection Plugin rearrangiert seine Inhalte, um sie im neuen Layout anzuzeigen.
\end{itemize}


