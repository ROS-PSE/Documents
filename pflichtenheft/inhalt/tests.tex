\newcounter{tfc}
\newcounter{tfcsec}
\newcounter{tmpsec}
\setcounter{tfc}{0}
\setcounter{tfcsec}{0}

\newcommand{\tf}{
	\addtocounter{tfc}{100}
	\setcounter{tfcsec}{0}
	
	\ifnum\value{tfc}<1000
		\item[/TF0\thetfc/]
	\else
		\item[/TF\thetfc/]
	\fi
}


\newcommand{\tfsec}
{
\addtocounter{tfcsec}{10}
\setcounter{tmpsec}{\thetfc}
\addtocounter{tmpsec}{\thetfcsec}
\ifnum\value{tfc}<1000
\item[/TF0\thetmpsec/]
\else
\item[/TF\thetmpsec/]
\fi
}


\chapter{Testf�lle und Testszenarien}
\section{Tests}
\begin{description}
	\tf rqt Plugin �ffnet und schlie�t ohne Fehler
	\tf Die GUI reagiert darauf, wenn ein Knoten keine Daten mehr versendet
	\tfsec , wenn ein Knoten langsamer oder schneller als sonst Daten sendet.
	\tfsec , wenn ein Knoten ma�geblich von seinen Sollwerten abweicht.
	\tf Eingelesene Konfigurationsdaten lassen sich parsen und auslesen
	\tf Hostdaten werden korrekt �bermittelt
	\tf rqt Plugin reagiert auf Gr��en�nderung
\end{description}

\section{Beispielszenarien}

\subsection*{Knoten exisitert nicht mehr}
\begin{itemize}
  \item Ein ROS-Netzwerk mit beliebig vielen Knoten wird gestartet
  \item Der Benutzer �ffnet das Plugin in der rqt GUI
  \item Nun wird mit "Gewalt" ein Knoten vom Netzwerk getrennt
  \item	Die GUI muss darauf reagieren und anzeigen, dass der Knoten nicht mehr
  verf�gbar ist und ein Fehler aufgetreten ist
\end{itemize}

\subsection*{Knoten weicht von seinen Sollwerten ab}
\begin{itemize}
  \item Wieder wird ein ROS-Netzwerk mit beliebig vielen Knoten gestartet
  \item Es werden f�r einen Knoten, der stetig Daten versendet, die Sollwerte so
  eingestellt, dass der Ist-Zustand weit von ihnen abweicht
  \item Der Benutzer �ffnet das Plugin in der rqt GUI
  \item Es muss deutlich angezeigt werden, dass der Knoten von den
  vordefinierten Sollwerten abweicht
\end{itemize}