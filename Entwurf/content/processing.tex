
\subsection{MonitoringNode}
Main Class wrapping the processing functionality.

\subsubsection{Attributes}
\begin{itemize}
	\item private MetadataStorage storage
	\item private SpecificationHandler specHandler
\end{itemize}
\subsubsection{Methods}
\begin{itemize}
	\item private Metadata receive_data()\\
	Receives data incoming from the Subscriber and converts them to Metadata objects.
	\item private ComparisonResult process_data(Metadata)\\
	Returns the specHandler's compare result
	\item private void publish_data(ComparisonResult)\\
	Publishes results of the comparison as rated Metadata
	\item private MetadataStorageResponse storage_server(MetadataStorageRequest)\\
	Listen for the GUI Model service calls and returns requested metadata from the storage
\end{itemize}


\subsection{MetadataStorage}
Saves recieved metadata packages for a given period of time and can provide them on request.

\subsubsection{Attributes}
\begin{itemize}
	\item private dict(string, dict(int, StorageContainer[])) storage\\
	Datastructure to store Packages by key and timestamp
	\item private int duration\\
	Duration in seconds for data to be stored
\end{itemize}
\subsubsection{Methods}
\begin{itemize}
	\item private void clean_up()\\
	Deletes Metadata exceeding the duration to store
	\item public boolean store(StorageContainer)\\
	Stores a given Metadata
	\item public StorageContainer[] get(string, int)\\
	Returns all Metadata packages for the given connection/host of the given amount of time.
	\item public boolean clear()\\
	Clears the whole storage
\end{itemize}


\subsection{StorageContainer}
Wraps Metadata in raw and rated form with an identifier and a timestamp. Object to be returned on request by the GUI model.

\subsubsection{Attributes}
\begin{itemize}
	\item public int timestamp\\
	Time when the data came from the subscriber
	\item public string identifier\\
	Host/Node/Connection identifier
	\item public object data\_raw\\
	The data as it reaches the subscriber from nodes and hosts.
	\item public object data\_rated\\
	The data like it would be published after being rated.
\end{itemize}


\subsection{Metadata}
Wraps metadata of exactly one host or node, a topic or a node-topic-combination.

\subsubsection{Attributes}
\begin{itemize}
	\item private MetadataTuple[] values\\
	Collection of Metadata regarding multiple meassurements.
\end{itemize}
\subsubsection{Methods}
\begin{itemize}
	\item public void add_tuple(MetadataTuple)\\
	Add a MetadataTuple of information to the bundle.
	\item public object get(String)\\
	Returns the value of the MetadataTuple with the given key. False, if the key does not exist.
\end{itemize}


\subsection{Specification}
TODO: Description

\subsubsection{Attributes}
\begin{itemize}
	\item private MetadataTuple[] values\\
	Collection of MetadataTuple objects providing limits for multiple fields.
\end{itemize}
\subsubsection{Methods}
\begin{itemize}
	\item public void add_tuple(MetadataTuple)\\
	Adds a MetadataTuple to the bundle
	\item public Object get(String)\\
	Returns the value of the MetadataTuple with the given key. The returned value would be a list containing limit values for the most measured fields. False, if the key does not exist.
\end{itemize}


\subsection{SpecificationHandler}
Loads the specifications from the parameter server and compares them to the actual metadata.

\subsubsection{Attributes}
\begin{itemize}
	\item private Specification[] specifications\\
	Datastructure to keep all loaded Specification objects
\end{itemize}
\subsubsection{Methods}
\begin{itemize}
	\item private void load_specifications()\\
	Loads the specifications from configuration files into Specification objects and stores them
	\item public RatedStatistics compare(Metadata, Specification)\\
	Compares a given Metadata object with a given Specification object regarding all available fields. Returns a RatedStatistics object wrapping potential divergences.
\end{itemize}


\subsection{RatedStatistics}
Wraps the result of the comparison between the actual metadata and the specificaton.

\subsubsection{Attributes}
\begin{itemize}
	\item public string seuid\\
	Identifies the node/host/connection
	\item public string[] metatype\\
	The metadata that was out of bounds
	\item public string[] actual\\
	The actual values
	\item public string[] expected\\
	The expected values
	\item public int[] state\\
	State of the metadata from the node/host/connection : state: { 0 = high ; 1 = low ; 2 = unknown}
\end{itemize}


\subsection{MetadataTuple}
Stores any kind of value for a certain key. Specifications storing values indicating limits, Metadata storing absolute actual values.

\subsubsection{Attributes}
\begin{itemize}
	\item private String key
	\item private Object value
\end{itemize}
\subsubsection{Methods}
\begin{itemize}
	\item public String get_dey()
	\item public Object get_value()
\end{itemize}