\section{GUI - Package Structure}
\begin{figure}[!ht]
\begin{center}
\includegraphics[scale=1.0]{./bilder/package_structure_gui.png}
\caption{The package structure of the GUI}
\label{The package structure of the GUI}
\end{center}
\end{figure}

\mbox{}

\newpage


\section{GUI - Model}
\begin{figure}[!ht]
\begin{center}
\includegraphics[width=1.0\linewidth]{./bilder/model.png}
\caption{The model class diagram}
\end{center}
\end{figure}

\subsection{BufferThread}
This thread should buffer the incoming data and regulary update the model and
hence also the model.
\subsubsection{Attributes}
\begin{itemize}
  \item private buffer\_lock: threading.Lock\\
  ..
  \item private model: ROSModel\\
  ..
  \item private timer: rospy.Timer\\
  ..
  \item private buffer: list<TimeStampedData>\\
  ..
\end{itemize}
\subsubsection{Methods}
\begin{itemize}
  \item public \_\_init\_\_()\\
  ..
  \item public start()\\
  ..
  \item public update\_model()\\
  ..
  \item public add\_buffer\_ite(object)\\
  ..
\end{itemize}

\subsection{ROSModel}
Represents the data as a QtModel. This enables automated updates of the View.
\subsubsection{Attributes}
\begin{itemize}
  \item \textit{private item\_list: list}
  ..
  \item \textit{private monitoring\_proxy: rospy.ServiceProxy}
  ..
  \item \textit{private host\_proxy: rospy.ServiceProxy}
  ..  
  \item \textit{private model\_lock: threading.Lock}
  ..
\end{itemize}
\subsubsection{Methods}
\begin{itemize}
  \item \textit{public \_\_init\_\_()}
  ..
  \item \textit{public QVariant data(QModelIndex, int)}
  ..
  \item \textit{public ItemFlags flags(const QModelIndex)}
  ..
  \item \textit{public QVariant headerData(int, Orientation, int)}
  ..
  \item \textit{public QModelIndex index(int, int, QModelIndex)}
  ..
  \item \textit{public QModelIndex parent(QModelIndex)}
  ..
  \item \textit{public int rowCount(QModelIndex)}
  ..
  \item \textit{public int columnCount(QModelIndex)}
  ..
  \item \textit{public update\_model(data:list)}
  ..
  \item \textit{public transform\_data(AbstractItem)}
  ..
  \item \textit{public transform\_data(statistics)}
  ..
  \item \textit{public transform\_data(statistics/host)}
  ..
  \item \textit{public transform\_data(statistics/rated)}
  ..
  \end{itemize}

\subsection{AbstractItem}
Provides a unified interface to access the items of a model.
\subsubsection{Attributes}
\begin{itemize}
  \item \textit{private data\_list: list<TimeStampedData>}
  ..
  \item \textit{private child\_items: list<AbstractItem>}
  ..
  \item \textit{private parent\_item: AbstractItem}
  ..
\end{itemize}
\subsubsection{Methods}
\begin{itemize}
  \item \textit{public \_\_init(list, parent=None)\_\_}
  ..
  \item \textit{public append\_child(AbstractItem)}
  ..
  \item \textit{public append\_data(TimeStampedData)}
  ..
  \item \textit{public AbstractItem get\_child(int)}
  ..
  \item \textit{public TimeStampedData get\_latest\_data()}
  ..
  \item \textit{public AbstractItem parent()}
  ..
  \item \textit{public AbstractItem get\_oldest\_item()}
  ..
  \item \textit{public delete\_oldest\_item()}
  ..
  \item \textit{public abstract execute\_action(RemoteAction)}
  ..  
\end{itemize}

\subsection{HostItem}
A HostItem represents a host with all its data.
\subsubsection{Methods}
\begin{itemize}
  \item \textit{public \_\_init(list, parent=None)\_\_}
  ..
  \item \textit{public execute\_action(RemoteAction)}
  ..
\end{itemize}

\subsection{NodeItem}
 A NodeItem represents a node with all of its data. It also has a interface to start/stop/restart nodes.
\subsubsection{Methods}
\begin{itemize}
  \item \textit{public \_\_init(list, parent=None)\_\_}
  ..
  \item \textit{public execute\_action(RemoteAction)}
  ..
\end{itemize}

\subsection{TopicItem}
..
\subsubsection{Methods}
\begin{itemize}
  \item \textit{public \_\_init(list, parent=None)\_\_}
  ..
  \item \textit{public execute\_action(RemoteAction)}
  ..
\end{itemize}

\subsection{ConnectionItem}
 A NodeItem represents a node with all of its data. It also has a interface to start/stop/restart nodes.
\subsubsection{Methods}
\begin{itemize}
  \item \textit{public \_\_init(list, parent=None)\_\_}
  ..
  \item \textit{public execute\_action(RemoteAction)}
  ..
\end{itemize}

\subsection{TimeStampedData}
..
\subsubsection{Attributes}
\begin{itemize}
  \item \textit{private data: object}
  ..
  \item \textit{private time\_stamp: datetime.time}
  ..
\end{itemize}
\subsubsection{Methods}
\begin{itemize}
  \item \textit{public \_\_init(object, datetime.time)\_\_}
  ..
  \item \textit{public datetime.time get\_time\_stamp()}
  ..
  \item \textit{public object get\_data()}
  ..
\end{itemize}

\subsection{Enum RemoteAction}
TODO: Description
\subsubsection{Types}
\begin{itemize}
	\item \textit{e\_action\_stop}
	..
	\item \textit{e\_action\_restart}
	..
\end{itemize}

\newpage
\section{GUI - View}
\begin{figure}[!ht]
\begin{center}
\includegraphics[width=0.8\linewidth]{./bilder/view.png}
\caption{The view class diagram}
\end{center}
\end{figure}

\subsection{OverviewPlugin}
The class OverviewPlugin is the core of the graphical user interface, which
contains most of the relevant information in a small and fancy area.
\subsubsection{Attributes}
\begin{itemize}
  \item \textit{public overview\_widget: QWidget}
  the object wich holds the widget
  \item \textit{public status\_light\_label: QLabel}
  shows the actual status of the host/node
  \item \textit{public tab\_widget: QTabWidget}
  the object wich holds the different tabs of the widget
  \item \textit{public information\_tab: QWidget}
  a tab wich gives general information about the network 
  \item \textit{public graphs\_tab: QWidget}
  shows the actual Network- and CPU-Load
  \item \textit{public range\_combo\_box: QComboBox}
  makes it possible to set the range of the graphs
  \item \textit{public log\_tab: QWidget}
  shows actual errors and warnings
  \item \textit{public log\_tab\_tree\_widget: QTreeWidget}
  the structure of the log-messages  
\end{itemize}

\subsection{SelectionPlugin}
A class which shows detailed information in a Tree-Layout about the currently
selected host or node.
\subsubsection{Attributes}
\begin{itemize}
  \item \textit{public selection\_widget: QWidget}
  the object wich holds the widget
  \item \textit{public host\_node\_label: QLabel}
  shows the name of the actual selected host or node
  \item \textit{public status\_light\_label: QLabel}
  shows the status of the host or node
  \item \textit{public tab\_widget: QTabWidget}
  the object wich holds the different tabs of the widget
  \item \textit{public information\_tab: QWidget}
  a tab wich gives general information about host or node 
  \item \textit{public graphs\_tab: QWidget}
  shows the actual Network- and CPU-Load
  \item \textit{public range\_combo\_box: QComboBox}
  makes it possible to set the range of the graphs
  \item \textit{public log\_tab: QWidget}
  shows actual errors and warnings
  \item \textit{public actions\_tab: QWidget}
  includes buttons to start/restart/stop nodes  
\end{itemize}

\subsection{TreePlugin}
TreePlugin is very simply and shows only the actual active hosts
and nodes.
\subsubsection{Attributes}
\begin{itemize}
  \item \textit{public tree\_widget: QWidget}
  the object wich holds the widget
  \item \textit{public erroneous\_check\_box: QCheckBox}
  displays only erroneous hosts and nodes
  \item \textit{public show\_node\_check\_box: QCheckBox}
  displays nodes
  \item \textit{public show\_host\_check\_box: QCheckBox}
  ..hosts
  \item \textit{public minus\_push\_button: QPushButton}
  makes it possible to zoom out
  \item \textit{public plus\_push\_button: QPushButton}
  ..zoom in
  \item \textit{public filter\_line\_edit: QLineEdit}
  a textfield where you can filter the output
\end{itemize}

