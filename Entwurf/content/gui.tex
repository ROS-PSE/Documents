\section{GUI - Package Structure}
\begin{figure}[!ht]
\begin{center}
\includegraphics[scale=1.0]{./bilder/package_structure_gui.png}
\caption{The package structure of the GUI}
\label{The package structure of the GUI}
\end{center}
\end{figure}

\mbox{}

\newpage


\section{GUI - Model}
\begin{figure}[!ht]
\begin{center}
\includegraphics[width=1.0\linewidth]{./bilder/model.png}
\caption{The model class diagram}
\end{center}
\end{figure}

\subsection{BufferThread}
This thread should buffer the incoming data from the topics and regulary update the model.
\subsubsection{Attributes}
\begin{itemize}
  \item \textit{private buffer\_lock: threading.Lock} the lock that guards the buffer from getting modified parallely
  \item \textit{private model: ROSModel} the model of the hosts/nodes/topics/connections 
  \item \textit{private timer: rospy.Timer} ROS Timer which regularily calls update\_model(). Btw Micha ist ...
  \item \textit{private buffer: list<TimeStampedData>} buffers the tons of incomming data by simply storing it here together with a timestamp for later usage
\end{itemize}
\subsubsection{Methods}
\begin{itemize}
%  \item \textit{public \_\_init\_\_()} 
  \item \textit{public start()} starts the thread and also the timer for regulary updates of the model
  \item \textit{public update\_model()} starts the update of the model. Will be called regulary by the timer.
  \item \textit{public add\_buffer\_item(object)} adds an item to the buffer list. Will be called whenever data from the topics is avaiable.
\end{itemize}

\subsection{ROSModel}
Represents the data as a QtModel. This enables automated updates of the View.
\subsubsection{Attributes}
\begin{itemize}
 % \item \textit{private item\_list: list}
  \item \textit{private log: list<TimeStampedData<list<string>>>} global log of all the events having occured in the past
  \item \textit{private monitoring\_proxy: rospy.ServiceProxy} the proxy to the monitoring node for obtaining statistics and rated statistics of the past minutes. To be called only once when the GUI started and the MonitoringNode has been running for a while 
  \item \textit{private host\_proxy: rospy.ServiceProxy} the proxy to a host for sending signals like stop or restart
  \item \textit{private model\_lock: threading.Lock} protects the model from parallel modification
\end{itemize}
\subsubsection{Methods}
\begin{itemize}
%  \item \textit{public \_\_init\_\_()}
  \item \textit{public object data(QModelIndex, int)}
  \item \textit{public ItemFlags flags(const QModelIndex)}
  \item \textit{public object headerData(int, Orientation, int)}
  \item \textit{public QModelIndex index(int, int, QModelIndex)}
  \item \textit{public QModelIndex parent(QModelIndex)}
  \item \textit{public int rowCount(QModelIndex)}
  \item \textit{public int columnCount(QModelIndex)}
  \item \textit{public update\_model(data:list)}
  \item \textit{public transform\_data(AbstractItem)}
  \item \textit{public transform\_data(statistics)}
  \item \textit{public transform\_data(statistics/host)}
  \item \textit{public transform\_data(statistics/rated)}
  \end{itemize}

\subsection{AbstractItem}
Provides a unified interface to access the items of a model.
\subsubsection{Attributes}
\begin{itemize}
  \item \textit{private data\_list: list<TimeStampedData>} contains the data of the abstract item represented as TimeStampedItems so that the progress in time can be shown
  \item \textit{private child\_items: list<AbstractItem>} the child of this item
  \item \textit{private parent\_item: AbstractItem} the parent of this item
\end{itemize}
\subsubsection{Methods}
\begin{itemize}
%  \item \textit{public \_\_init(list, parent=None)\_\_}
   \item \textit{public append\_child(AbstractItem)} append a child to the list of childs
  \item \textit{public append\_data(TimeStampedData)} append data to the data\_list of the AbstractItem
  \item \textit{public AbstractItem get\_child(row: int)} return the child at the position row
  \item \textit{public TimeStampedData get\_latest\_data()} return the latest data of the data\_list
  \item \textit{public AbstractItem parent()} returns the parent of this or None if there is none (TODO:is that correct)
  \item \textit{public AbstractItem get\_oldest\_item()} return the oldest data item in data\_list
  \item \textit{public delete\_oldest\_item((age:int))} delets the oldest item in data\_list. Optional if age is set: if the age of the oldest data item is greater equal to the age definded it will be deleted
  \item \textit{public abstract execute\_action(RemoteAction)} executes a action on the current item like stop or restart. Calls to this method should be redirected to the remote host on executed there.
\end{itemize}

\subsection{HostItem}
A HostItem represents a host with all its data.
The data\_list will contain TimeStampedData whereas the data is a dict containing ...
\subsubsection{Methods}
\begin{itemize}
%  \item \textit{public \_\_init(list, parent=None)\_\_}
  \item \textit{public execute\_action(RemoteAction)}
  ..
\end{itemize}

\subsection{NodeItem}
 A NodeItem represents a node with all of its data. It also has a interface to start/stop/restart nodes.
 The data\_list will contain TimeStampedData whereas the data is a dict containing ...
\subsubsection{Methods}
\begin{itemize}
 % \item \textit{public \_\_init(list, parent=None)\_\_}
  \item \textit{public execute\_action(RemoteAction)}
\end{itemize}

\subsection{TopicItem}
A ConnectionItem reprensents the connection between a publisher and a subscriber and the topic they are puglishing / listenening on.
The data\_list will contain TimeStampedData whereas the data is a dict containing ...
\subsubsection{Methods}
\begin{itemize}
%  \item \textit{public \_\_init(list, parent=None)\_\_}
  \item \textit{public execute\_action(RemoteAction)} not senseful, throws an exception
\end{itemize}

\subsection{ConnectionItem}
 A NodeItem represents a node with all of its data. It also has a interface to start/stop/restart nodes.
 The data\_list will contain TimeStampedData whereas the data is a dict containing ...
\subsubsection{Methods}
\begin{itemize}
%  \item \textit{public \_\_init(list, parent=None)\_\_}
  \item \textit{public execute\_action(RemoteAction)} not senseful, throws an exception
\end{itemize}

\subsection{TimeStampedData}
..
\subsubsection{Attributes}
\begin{itemize}
  \item \textit{private data: object}
  ..
  \item \textit{private time\_stamp: datetime.time}
  ..
\end{itemize}
\subsubsection{Methods}
\begin{itemize}
  \item \textit{public \_\_init(object, datetime.time)\_\_}
  ..
  \item \textit{public datetime.time get\_time\_stamp()}
  ..
  \item \textit{public object get\_data()}
  ..
\end{itemize}

\subsection{Enum RemoteAction}
TODO: Description
\subsubsection{Types}
\begin{itemize}
	\item \textit{e\_action\_stop}
	..
	\item \textit{e\_action\_restart}
	..
\end{itemize}

\newpage
\section{GUI - View}
\begin{figure}[!ht]
\begin{center}
\includegraphics[width=0.8\linewidth]{./bilder/view.png}
\caption{The view class diagram}
\end{center}
\end{figure}

\subsection{OverviewPlugin}
The OverviewPlugin is the core of the graphical user interface, which
contains most of the relevant information in a small and fancy area.
\subsubsection{Attributes}
\begin{itemize}
  \item \textit{public overview\_widget: QWidget}\\
  the object wich holds the widget
  \item \textit{public status\_light\_label: QLabel}\\
  a status ligth wich shows if everything is ok or not
  \item \textit{public tab\_widget: QTabWidget}\\
  the object wich holds the different tabs of the widget
  \item \textit{public information\_tab: QWidget}\\
  a tab wich gives general information about the network 
  \item \textit{public graphs\_tab: QWidget}\\
  displays graphs about the network
  \item \textit{public range\_combo\_box: QComboBox}\\
  makes it possible to set the range of the graphs
  \item \textit{public log\_tab: QWidget}\\
  shows actual errors and warnings 
\end{itemize}

\subsection{SelectionPlugin}
A Plugin which shows detailed information in different tabs about the currently
selected host or node.
\subsubsection{Attributes}
\begin{itemize}
  \item \textit{public selection\_widget: QWidget}\\
  the object wich holds the widget
  \item \textit{public host\_node\_label: QLabel}\\
  the name of the actual selected host or node
  \item \textit{public status\_light\_label: QLabel}\\
  a status-light about the status of the current host or node
  \item \textit{public tab\_widget: QTabWidget}\\
  the object wich holds the different tabs of the widget
  \item \textit{public information\_tab: QWidget}\\
  a tab wich gives general information about hosts or nodes 
  \item \textit{public graphs\_tab: QWidget}\\
  displays graphs about the actual selected host or node, e.g the Network- and
  CPU-Load
  \item \textit{public range\_combo\_box: QComboBox}\\
  makes it possible to set the range of the graphs
  \item \textit{public log\_tab: QWidget}\\
  shows actual errors and warnings
  \item \textit{public actions\_tab: QWidget}\\
  includes buttons to restart and stop nodes  
\end{itemize}

\subsection{TreePlugin}
TreePlugin is very simply and shows only the actual active hosts
and nodes. It is possible to filter the output, e.g. only erroneus hosts or
nodes are displayed.
\subsubsection{Attributes}
\begin{itemize}
  \item \textit{public tree\_widget: QWidget}\\
  the object wich holds the widget
  \item \textit{public erroneous\_check\_box: QCheckBox}\\
  only erroneous hosts and nodes will be displayed
  \item \textit{public show\_node\_check\_box: QCheckBox}\\
  displays the activ nodes
  \item \textit{public show\_host\_check\_box: QCheckBox}\\
  displays the activ hosts
  \item \textit{public plus\_push\_button: QPushButton}\\
  makes it for, a better clarity, possible to zoom in  
  \item \textit{public minus\_push\_button: QPushButton}\\
  and zoom out
  \item \textit{public filter\_line\_edit: QLineEdit}\\
  a textfield where you can define a filter for the output
\end{itemize}

